% !TEX encoding = UTF-8 Unicode
%\documentclass[english,onecolumn]{IEEEtran}
\documentclass[a4paper]{scrartcl}
\usepackage[a4paper,hmargin=2.5cm]{geometry}
\usepackage[toc,page]{appendix}

\usepackage{times}
\usepackage[T1]{fontenc}
\usepackage[utf8]{inputenc}
\usepackage{array}
\usepackage{multirow}
\usepackage{float}
\usepackage{url}
\usepackage{amsthm}
\usepackage{amsmath}
\usepackage{amssymb}
\usepackage{graphicx}
\usepackage{color}
\usepackage{caption}
\usepackage[english]{babel}




\newcommand{\nati}[1]{{\textcolor[rgb]{.1,.6,.1}{#1}}}



\begin{document}

\title{Summer School Proposal}
\subtitle{Key Insights in Networks and Graphs 2016}

\author{Nathanael Perraudin, Johan Paratte, Lionel Martin and Michael Defferrard}

\maketitle


% \begin{abstract}
% \end{abstract}

\vspace{2cm}


{\bf{Dear Sir or Madam,}} \\
we hereby wish to apply for funding to organize the second edition of a
\begin{large}
\begin{center}
Summer School on Graphs and Networks
\end{center}
\end{large}
as a project supported by the LTS2 Group at EPFL. We have so far formed a committee of four members, namely:
\begin{itemize}
	\item Nathanaël Perraudin,
	\item Johan Paratte, 
	\item Lionel Martin,
	\item Michael Defferrard.
\end{itemize}

This committee works in collaborations with the following professors
\begin{itemize}
	\item Pierre Vandergheynst (LTS2 - EPFL),
	\item Pascal Frossard (LTS4 - EPFL),
	\item Jean-Philippe Thiran (LTS5 - EPFL),
	\item ???
\end{itemize}

In the Summer School we would like to highlight state-of-the-art and common graph signal processing methods. This reflects in a broad range of speakers that we consider to invite. We are planning of inviting four different speakers: 1 from Switzerland, 1 from Europe and 2 from the US.

\vspace{0.5cm}

Thank you for considering our proposal. Kind regards, the organizing committee

\vspace{0.5cm}

Johan Paratte, Barbara Keller, Lionel Martin, Michael Defferrard









\newpage

\section{Previous edition}
\nati{TO BE DONE}

\section{Speakers}

\paragraph{Confirmed speakers}
\begin{itemize}
	\item Pierre Vandergheynst : Signal processing on graphs
	\item Mauro Maggioni : The mathematics behind the scene of graph signal processing.
\end{itemize}

\paragraph{Potential speakers}
\begin{itemize}
	\item Mikhail Belkin : Graphs in machine and human learning
	\item Daniel Spielman : Graph partitioning, sparsification
	\item Fan Chung Graham : Graph theory in the information age
\end{itemize}

\section{Target Group}
This Summer School will gather people from different fields using similar tools and techniques, for example from signal processing, graph analysis and machine learning. We deem the Summer School to be of interest for Ph.D. students and postdoctoral researchers from the following EPFL laboratories:
\begin{center}
LTS2, LTS4, LTS5, LTHI, LTHC
\end{center}
We expect not only swiss participants to attend the Summer School and we are also open to the inscription of external people, for example from the following research groups with which we collaborate:
\begin{itemize}
	\item ENS Lyon (Pierre Borgnat)
	\item University of southern California (Antonio Ortega)
	\item INRIA Rennes (Rémi Gribonval)
	\item Carnegie Mellon University Pittsburgh (José Moura)
	\item University of California, Berkeley (Kannan Ramchandran)
	\item Télécom Bretagne (Vincent Gripon)
\end{itemize}


\section{ECTS Points}
We suggest to offer 2 ECTS points to Ph.D. students successfully participating in the Summer School. A certificate of participation will be delivered at the end of the summer school. The university will then decide if they accept to deliver the credits.
%To that end, we would like to test the student’s comprehension of the presented topics. This will be accomplished by handing out an exercise sheet covering the topics in question. In addition to that, the exercise sheet will also award points for good suggestions where the newly learned techniques could be applied in their field of study. We believe this will give students the opportunity to lay the foundation for fruitful collaboration. The evaluation of the handed in solutions will be performed at the end of the summer school.

\section{Location}
With the idea that nice places can help open the mind, we propose to hold this Summer School in Leukerbad, a small town in the canton of Valais up in a valley in the Alps. It is surrounded by mountains and offers a nice panorama. Leukerbad is the largest thermal spa resort in the Alps. The region also offers hiking tours and has cable car access to Torrent or Gemmi pass. Leukerbad is thus a place which is quiet for the mind and relaxing for the body. 

\section{Schedule}
Each of the four speakers will be in charge of six lectures, split into one morning and one afternoon session, preferably on different days. We plan to offer a small excursion on Wednesday afternoon. This leaves us with four full days for lectures.

\vspace{0.5cm}
{{\bf{Typical day Schedule}}\\
\begin{tabular}{lll}
8:00 - 8:45 &:& Breakfast \\
9:00 - 9:45 &:& Lesson 1 \\
10:00 - 10:45 &:&  Lesson 2 \\
10:45 - 11:15 &:&  Break // coffee \\
11:15 - 12:00 &:&  Lesson 3 \\
12:15 - 13:15 &:&  Lunch \\
14:00 - 14:45 &:&  Lesson 4 \\
15:00 - 15:45 &:&  Lesson 5 \\
15:45 - 16:15 &:&  Break // coffee \\
16:15 - 17:00 &:&  Lesson 6 \\
17:15 - 19:00 &:&  Free time, discussions \\
20:00 &:&  Dinner and social events
\end{tabular}

\section{Dates}
The dates selected for the summer school are from June 22nd to June 26th. 


\section{Budget}
The minimum number of participants is 20 and the maximum 30. If each Ph. D. participant is asked to pay a fee of $450$ CHF ($650$ for posdoctoral researchers and $1000$ for non academic participant), the budget and associated requested financial support is as follows.

We have counted 4 nights per participant since most of them can travel back on Friday night. However, for the speakers, we have counted 5 nights each since they wont be able to travel before Saturday. Participants will share four-bed rooms (they can pay extra money to sleep in double or single bedroom) and the speakers will have a room for themselves. This budget includes all meals of the week.

A detail of the cost of a participant is presented in the table bellow.
% \ref{tab:budget}.
%\begin{table} \label{tab:budget}
\begin{center}
\begin{tabular}{|l|r|r|r|}
\hline
Description &	Amount (CHF) & Quantity & Cost (CHF) \\
\hline
Food + Accommodation (participant) & 550 & 30 & 16500 \\
Food + Accommodation (speaker) & 800 & 4 & 3200 \\
Conference room & ??? & 1 & ??? \\
Coffee breaks & ??? & 272 &	??? \\
Travel cost for speaker in Switzerland & 200 & 1 & 200 \\
Travel cost for speaker abroad (EU) & 800 & 1 & 800 \\
Travel cost for speaker abroad (US)	& 2000 & 2 & 4000 \\
Excursion & 30 & 34 & 1020 \\
Varia & 1500 & 1 & 1500 \\
\hline
\multicolumn{3}{|l|}{Total of the costs} & ??? \\
\hline					
Attendees financial participation &	550	& 30 & 16500 \\
\hline	
\multicolumn{3}{|l|}{{\bf{Requested financial support}}} & {\color{red}????} \\
\hline
\end{tabular}
\end{center}
%\end{table}


\end{document}
