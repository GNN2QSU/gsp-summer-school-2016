%\documentclass[english,onecolumn]{IEEEtran}
\documentclass[a4paper]{scrartcl}
\usepackage[utf8]{inputenc} \usepackage[T1]{fontenc}
\usepackage{lmodern,textcomp}
\usepackage[english]{babel}
\usepackage[hmargin=2.5cm]{geometry}
\usepackage[toc,page]{appendix}
%\usepackage{times}
%\usepackage{array}
%\usepackage{multirow}
\usepackage{float}
\usepackage{url}
%\usepackage{amsthm}
\usepackage{amsmath}
\usepackage{amssymb}
%\usepackage{graphicx}
\usepackage{color}
%\usepackage{caption}
\usepackage[style=numeric,backend=biber]{biblatex} \bibliography{refs}
\usepackage[colorlinks=true,allcolors=blue]{hyperref}

\begin{document}

\title{Graph Signal Processing Summer School}
\subtitle{Proposal for a research school at CIRM during the first semester 2018}
\author{
	Michaël Defferrard \and
	Nathanaël Perraudin \and
	Yann Schoenenberger \and
	Dorina Thanou
	%Lionel Martin \and
	%Johan Paratte
}
\maketitle

\begin{abstract}
	Proposal for a Summer School on Graph Signal Processing to be held in 2018
	at CIRM, Marseille.
\end{abstract}

\section{Summary}

The proposed Summer School aims to provide PhD students and researchers with
insights on advanced tools and methodologies in Graph Signal Processing (GSP)
research. The goal is to disseminate state-of-the-art research about signal
processing on graphs as well as discussing and defining new directions both on
the theoretical developments of the GSP framework and on its applications to
real science and engineering problems.
% Two levels
There will be a set of introductory level lectures for those interested to apply
GSP tools to their research or problems as well as a set of advanced topics
geared towards researchers in the field. The exact topics have yet to be
discussed with potential speakers. They will offer a mix of theoretical and
applied talks.

Summer school participants will have the opportunity to learn and study
innovative algorithms and systems in the signal processing field, in an
interactive and stimulating way: each day, after the lectures, the speakers will
be available for face-to-face discussion with participants who are interested in
deepening their knowledge of the covered material. Students will have the
possibility to present their work to let other participants and lecturers know
their research activity. There will be free time for students to discuss,
exchange ideas, brainstorm or engage in debates.

The first edition of this school, previously named \textit{Key Insights on
Networks and Graphs}\footnote{\url{https://lts2.epfl.ch/summerschool}}, was
held in Leukerbad, Switzerland, in summer 2015.

\section{Résumé}

La même chose en français.


\section{Scientific content}

% Graphs == avenir du futur
With the advent of Big Data, an increasing amount of data is becoming available
and advanced models are needed to extract meaningful information from data and
identify relational structures that facilitate different data analysis tasks
such as recognition or inference. Graph Signal Processing (GSP) enables the
analysis and processing of signals that reside on irregular (non-Euclidean)
domains, such as infrastructure or social networks. Graphs may further be used
to incorporate external information into data models, such as hyperlinks or
citations between documents, and to capture the underlying geometrically complex
manifold structure of the data. 

% Graphs
Graphs are generic data representation forms that are useful for describing the
geometric structures of data domains in numerous applications, including social,
energy, transportation, sensor, and neuronal networks.
% Weights
The weight associated with each edge in the graph often represents the
similarity between the two vertices it connects. The connectivities and edge
weights are either dictated by the physics of the problem at hand or inferred
from the data. For instance, the edge weight may be inversely proportional to
the physical distance between nodes in the network.
% Signals
The data on these graphs can be visualized as a finite collection of samples,
with one sample at each vertex in the graph. Collectively, we refer to these
samples as a graph signal. 

% GSP
The purpose of GSP is to exploit the underlying structure to analyze, process
and learn signals residing on graphs. 
%Examples from the classical signal processing toolbox
Classical GSP tasks
include filtering, denoising, inpainting, compressing. An important
consideration when designing such tools is their computational complexity, which
can often be improved by spectral relaxations or numerical approximations.
Efficiency is key to cope with the ever growing size of datasets.

% summer school on both applications and theoretical advancements

% Feature vs data graph
There exists two paradigms to model the underlying structure of datasets using graphs: 1) constructing a
\textit{feature graph}, where nodes represent features and samples are graph
signals, or 2) a \textit{data graph}, where nodes represent samples and graph
signals are functions of data. In the first case the data is transformed using
GSP tools, while in the second case graphs are often used to regularize
classification or regression functions.  We then identify two regimes: 1) when
the graph structure is given by the application, and 2) when the graph has to be
constructed from features or data.
% math: g(L) to filter, Dirichlet regularization
% Applications: given feature graph
Examples of applications where a feature graph structure is given are found in
many different engineering and science fields. In transportation networks, we
may be interested in analyzing epidemiological data describing the spread of
diseases, census data describing human migration patterns, or logistics data
describing inventories of trade goods. Other examples include gene expression
patterns defined on top of gene networks, the congestion level at the nodes of a
telecommunication network, and patterns of brain activity defined on top of a
brain network.
% Applications: constructed feature graph
% Applications: constructed data graph
% Other examples: graph cuts
In statistical learning problems, GSP is a meaningful paradigm even when the
graph structure is not given by the problem but might be inferred from data.  In
such cases, the graph is a discrete approximation of a manifold embedded in an
Euclidean space: each node represents a sample and the weights represent
similarities between samples. Signals on those graphs may be the output of a
classification or regression model which we could constrain to be smooth or
piece-wise constant on the data graph. Applications include machine vision (e.g.
semi-local graphs between pixels) and automatic text classification.
Graph-based methods are especially popular for the (transductive or inductive)
semi-supervised learning problem where the objective is to classify unknown data
with the help of a few labeled samples and a large pool of unlabeled training
samples.
% Applications: given data graph
As an example of intrinsic data graph, instead of defining weights as
similarities between vector representations of documents (from e.g. the
bag-of-word model), external information might be incorporated by defining
binary weights as hyperlinks or citations between documents.

% Theoretical understanding
Besides particular applications, the school will also showcase the advancement
of the understanding of network data by redesigning traditional tools originally
conceived to study signals defined on regular domains (such as time-varying
signals or spatially varying images and fields) and extending them to analyze
signals on the more complex graph domain. Examples of topics to be showcased in
this theoretical track include graph transforms, sampling theorems, filter
design and uncertainty principles.

% Relevance
The last few years have seen significant progress in the development of theory,
tools, and applications of GSP. This school is intended to disseminate ideas to
a broader audience and to exchange ideas and experiences on the future path of
this emerging field. As the interest grows, a summer school has been organized
in 2015 in Switzerland and a
workshop\footnote{\url{https://alliance.seas.upenn.edu/~gsp16/wiki}} will take
place this year in the US. Last year, Pascal Frossard got invited to give a
plenary talk in the Sampling Theory and Applications (SampTA)
conference\footnote{\url{https://www.american.edu/cas/mathstat/sampta2015}} as
well as an overview talk about Multimedia graph-based processing at the
Conference on Multimedia and Expo
(ICME)\footnote{\url{http://www.icme2015.ieee-icme.org}}, which shows the
interest of the multimedia community in GSP.  While mainly focusing on complex
and dynamical networks, the \textit{Network Science Thematic
Semester}\footnote{\url{https://project.inria.fr/netspringlyon}}, which
comprises two workshops and a conference that will take place in May and June
2016 in Lyon and Marseille, France, features some keynotes and presentations
from the GSP community, notably with the interventions of Pierre Vandergheynst,
José M.F. Moura and Alejandro Ribeiro. The IEEE Image, Video, and
Multidimensional Signal Processing (IVMSP) workshop
2016\footnote{\url{http://www.ivmsp2016.org}}, focused on image and video
processing, mentions GSP and manifold modelings in its call for papers while
Pierre Vandergheynst will give a plenary talk on signal and image processing on
graphs.

% Our summer school
Please refer to published review papers \cite{shuman_emerging_2013,
sandryhaila_discrete_2014, coifman_diffusion_2006, ekambaram_circulant_2013} for
a broad technical overview as well as a discussion of the challenges of GSP.
Below is a selection of applicable topics with selected (recent) references. The
selection of topics to be presented will be discussed with the speakers and will
depend on their interests.
\begin{itemize}
	\setlength{\itemsep}{0pt} \setlength{\parskip}{0pt}
	% From GSP2016
	\item Sampling and recovery of graph signals \cite{puy2015random,tsitsvero2015signals,anis2014towards}
	\item Graph filter and filter bank design \cite{hammond2011wavelets,narang_bior_filters,ekambaram_globalsip,leonardi_multislice,leonardi_fmri,shuman2013spectrum}
	\item Uncertainty principles and other fundamental limits \cite{perraudin2016global,agaskar_spie,pasdeloup}
	\item Graph signal transforms \cite{szlam,gavish,Maggioni_biorthogonal,coifman_diffusion_2006,lafon_coarse,narang_lifting_graphs}
	\item Graph filter identification \cite{thanou_TSP_2014}
	\item Graph construction and graph topology inference \cite{kalofolias2016learn,Dong:2014tj}
%	\item Prediction and learning in graphs
	\item Statistical graph signal processing \cite{perraudin2016stationary,zhang2015graph,gadde2015probabilistic}
%	\item Signals in high-order graphs
%	\item Non-linear graph signal processing
	\item Graph-based image and video processing \cite{elmoataz2008nonlocal,lezoray2008nonlocal,shahid2015fast}
	\item Graph-based 3D multimedia processing \cite{zhang2014fast,
		schoenenberger2015graph}
	\item Applications to neuroscience and other medical fields \cite{huang2015graph,hu2015spectral}
	\item Applications to economics and social networks
	% Nath: I do not have any references for this one.
	\item Applications to infrastructure networks such as communication, transportation, power networks \cite{mcgraw,jain2014big,borgnat2011shared}
	% Added by Michaël
	\item Learning graph embeddings \cite{von2008consistency,zhou2006learning,belkin2003laplacian}
	\item Machine Learning with Graph regularization \cite{smola,belkin2005towards,zhou2004regularization,zhou2005learning,belkin2006manifold}
%	\item Extracting features from graph signals
	% Added by Nathanael
	\item Graph coarsening, multi-scale graph signal processing \cite{spielman2011graph,shuman2016multiscale,liu_coarsening,lafon_coarse,jansen}
	\item Graph spectral theory \cite{chung1997spectral,belkin2007convergence,shuman2015vertex,sandryhaila2014discrete,Nakatsukasa2013mysteries,brooks,dekel}
\end{itemize}

\section{Speakers}

We plan to invite five speakers to give lectures. We expect one of them, the
first to present on Monday, to introduce the field and give a motivational talk.
The purpose of this overview talk is to help novice attendants to follow the
more advanced talks. Speakers are selected such that all of those three
main tracks are covered:

\begin{enumerate}
	\setlength{\itemsep}{0pt} \setlength{\parskip}{0pt}
	\item Introduction to GSP
	\item Foundations of GSP
	\item Innovations and applications of GSP
	\item Optimization side-track
\end{enumerate} 

As there is a strong optimization community in France which could easily travel
to CIRM for a talk, we envision to introduce a ''side track`` about
optimization, which is quite relevant to GSP. We'll discuss with those speakers
on how to adapt their talks to our audience. We are very enthusiastic about the
impact this exchange could have. We however won't contact those speakers yet, as
it may not happen should this proposal be rejected (because the event may then
be located in another place with different local opportunities).

Below is the list of speakers we contacted to lecture at this school. As we
don't know yet the date and place of the school, we didn't asked them to confirm
their venue. While some did already pre-confirmed their attendance, other
confirmations are pending.
\begin{itemize}
\setlength{\itemsep}{0pt} \setlength{\parskip}{0pt}
\item TRACK 1
	\begin{itemize}
	\setlength{\itemsep}{0pt} \setlength{\parskip}{0pt}
	\item \href{http://people.epfl.ch/pierre.vandergheynst}
		{Pierre Vandergheynst}, EPFL, Switzerland
		(pre-confirmed)
		% Signal processing on graphs
	\end{itemize}
\item TRACK 2
	\begin{itemize}
	\setlength{\itemsep}{0pt} \setlength{\parskip}{0pt}
	\item \href{http://www.wsi.uni-tuebingen.de/lehrstuehle/theory-of-machine-learning/people/ulrike-von-luxburg.htmli}
		{Ulrike von Luxburg}, University of Tübingen, Germany
		(pending)
		% Nathanaël: spectral clustering
	\item \href{http://www.cs.yale.edu/homes/spielman/}
		{Daniel Spielman}, Yale University, USA
		(pending)
		% Graph partitioning, sparsification
	\end{itemize}
\item TRACK 3
	\begin{itemize}
	\setlength{\itemsep}{0pt} \setlength{\parskip}{0pt}
	\item \href{http://cs.stanford.edu/people/jure/}
		{Jure Leskovec}, Stanford University, USA
		(pending)
	\item Mauro, Elmoataz, Ortega, Gribonval, Borgnat, Kondor or Jelena ?
	\end{itemize}
\end{itemize}

In case the aforementioned speakers cannot make it, below is another list (in no
particular order) of potential ones. The diversity of this school is again
reflected in the broad range of considered speakers. Given this large list, we
are pretty confident that enough of them will be able to attend our event, given
the time left to advertise it.
\begin{itemize}
\setlength{\itemsep}{0pt} \setlength{\parskip}{0pt}
\item TRACK 1
	\begin{itemize}
	\setlength{\itemsep}{0pt} \setlength{\parskip}{0pt}
	\item \href{http://ee.usc.edu/faculty_staff/faculty_directory/ortega.htm}
		{Antonio Ortega}, University of Southern California, USA
		% GSP2016, Mauro: could talk a lot about a variety of applications
	\item \href{https://users.ece.cmu.edu/~moura/}
		{José M. F. Moura}, Carnegie Mellon University, USA
		% GSP2016
	\end{itemize}
\item TRACK 2
	\begin{itemize}
	\setlength{\itemsep}{0pt} \setlength{\parskip}{0pt}
	\item \href{http://web.cse.ohio-state.edu/~mbelkin/}
		{Mikhail Belkin}, Ohio State University, USA
		% Graphs in machine and human learning
	\item \href{http://math.ucsd.edu/~fan/}
		{Fan Chung Graham}, UC San Diego, USA
		% Graph theory in the information age
	\item \href{http://people.irisa.fr/Remi.Gribonval/}
		{Rémi Gribonval}, INRIA Rennes, France
		% Mauro: make connections to dictionary learning
	\end{itemize}
\item TRACK 3
	\begin{itemize}
	\setlength{\itemsep}{0pt} \setlength{\parskip}{0pt}
	\item \href{http://services.math.duke.edu/~mauro/}
		{Mauro Maggionni}, Duke University, USA
		% KING2015
	\item \href{https://elmoatazbill.users.greyc.fr/}
		{Abderrahim Elmoataz}, University of Caen Basse Normandie, France
		% suggested by Pascal
	\item \href{http://people.cs.uchicago.edu/~risi/}
		{Risi Kondor}, University of Chicago, USA
		% Mauro: multi-scale constructions in general
	\item \href{http://www.ece.mcgill.ca/~mrabba1/}
		{Michael Rabbat}, McGill University, Canada
		% GSP2016
	\item \href{http://perso.ens-lyon.fr/pierre.borgnat/}
		{Pierre Borgnat}, ENS Lyon / CNRS, France
		% suggested by Pascal
	\item \href{http://perso.ens-lyon.fr/patrick.flandrin/}
		{Patrick Flandrin}, ENS Lyon / CNRS, France
		% suggested by Pascal
	\item \href{http://infocom.uniroma1.it/sergio/}
		{Sergio Barbarossa}, Sapienza University of Rome, Italy
	\item \href{http://research.microsoft.com/en-us/people/pachou/}
		{Philip A. Chou}, Microsoft Research, USA
	\item \href{http://www.math.ucdavis.edu/~saito/}
		{Naoky Saito}, UC Davis, USA
	\item \href{http://alliance.seas.upenn.edu/~aribeiro/wiki/}
		{Alejandro Ribeiro}, University of Pennsylvania, USA
		% GSP2016
	\item \href{http://cpsc.yale.edu/people/ronald-coifman}
		{Ronald Coifman}, Yale University, USA
		% Xavier
	\item \href{http://www.commsp.ee.ic.ac.uk/~pld/}
		{Pier Luigi Dragotti}, Imperial College London, Great Britain
		% suggested by Mauro
	\item \href{http://jelena.ece.cmu.edu/}
		{Jelena Kovačević}, Carnegie Mellon University, USA
		% suggested by Mauro, GSP2016
		% Mauro: applications to biology
	\item \href{http://people.epfl.ch/pascal.frossard}
		{Pascal Frossard}, EPFL, Switzerland
		% suggested by Mauro, GSP2016
	\item \href{http://www.cpt.univ-mrs.fr/~barrat/english.html}
		{Alain Barrat}, Université Aix-Marseille / CNRS, France
		% Statistical physics, Benjamin peut le contacter
	\item \href{http://www.cs.cornell.edu/home/kleinber/}
		{Jon Kleinberg}, Cornell University, USA
%	\item \href{http://perso.ens-lyon.fr/nelly.pustelnik/}
%		{Nelly Pustelnik}, ENS Lyon / CNRS, France
		% Pascal: elle fait de l'optimization, récemment sur les graphes
	\end{itemize}
\item SIDE TRACK
	\begin{itemize}
	\setlength{\itemsep}{0pt} \setlength{\parskip}{0pt}
	\item \href{http://www.di.ens.fr/~fbach/}
		{Francis Bach}, INRIA Paris, France
		% suggested by Michaël
	\item \href{http://www.ljll.math.upmc.fr/~plc/}
		{Patrick Louis Combettes}, Université Pierre et Marie Curie, France
		% suggested by Nath
	\item \href{http://www-syscom.univ-mlv.fr/~pesquet/}
		{Jean-Christophe Pesquet}, Université Paris-Est, France
		% suggested by Nath, Pascal
	\item \href{http://fadili.users.greyc.fr/}
		{Jalal Fadili}, ENSI Caen, France
		% suggested by Pascal
	\item \href{http://gpeyre.github.io/}
		{Gabriel Peyré}, Université Paris-Dauphine / CNRS, France
		% suggested by Mauro
	\end{itemize}
\end{itemize}

\section{Participants}

This summer school targets an international audience, with a diverse set of
speakers and potentially interested research labs. It will gather people from
different fields using similar tools and techniques, for example from signal
processing, graph analysis and machine learning. If the proposal is accepted, we
will activate our networks and notify all the interested parties we know about.
In addition to the laboratories of the potential speakers, we expect the
students of the following groups to be interested, some of which we have
collaborated with.
\begin{itemize}
	\setlength{\itemsep}{0pt} \setlength{\parskip}{0pt}
	\item \href{https://www.eecs.berkeley.edu/~kannanr/}
		{Kannan Ramchandran}, UC Berkeley, USA
	\item \href{http://www.vincent-gripon.com/}
		{Vincent Gripon}, Télécom Bretagne, France
	\item \href{http://www.macalester.edu/~dshuman1/}
		{David Shuman}, Macalester College, USA
	\item \href{http://www.inf.usi.ch/bronstein/}
		{Michael M. Bronstein}, USI, Switzerland
	\item \href{http://camille.roth.free.fr/index.php}
		{Camille Roth}, Humboldt Universität / CNRS, Germany
	\item \href{http://xn.unamur.be/}
		{Renaud Lambiotte}, University of Namur, Belgium
	\item \href{https://web.eecs.umich.edu/~hero/}
		{Alfred Hero}, University of Michigan, USA
	\item \href{http://www.epicx-lab.com/vittoria-colizza.html}
		{Vittoria Colizza}, Inserm, France
	\item \href{http://people.epfl.ch/patrick.thiran}
		{Patrick Thiran}, EPFL, Switzerland
	\item \href{http://icapeople.epfl.ch/grossglauser/}
		{Matthias Grossglauser}, EPFL, Switzerland
\end{itemize}

We however do not expect more than 40 participants, the last edition attracted
32 students, such that we propose to co-locate it with the \textit{Dynamique des
Systèmes Biologiques} summer school, whose organizers anticipate around 40
participants too. As such we chose the same dates and will limit the number of
attendees to 40, such that all participants from the two schools can be
accommodated at CIRM. The lectures from the two events will take place in two
different auditoriums, namely at CIRM and CPPM\footnote{Confirmed by Bruno
Torrésani.}. While the scientific part will be organized independently, we plan
to jointly organize the social events (lunch, dinner, excursion) during which
students may exchange and eventually develop future collaborations.

\section{Others}

\subsection{Organization}

This school is to be organized by:
\begin{itemize}
	\setlength{\itemsep}{0pt} \setlength{\parskip}{0pt}
	\item Michaël Defferrard, PhD student, EPFL, Switzerland
	\item Nathanaël Perraudin, PhD student, EPFL, Switzerland
	\item Yann Schoenenberger, PhD student, EPFL, Switzerland
	\item Dorina Thanou, scientist, EPFL, Switzerland
\end{itemize}
and the scientific committee is composed of:
\begin{itemize}
	\setlength{\itemsep}{0pt} \setlength{\parskip}{0pt}
	%\item Pierre Vandergheynst, full professor, EPFL, Switzerland
	\item Pascal Frossard, associate professor, EPFL, Switzerland
	\item Bruno Torrésani, full professor, Aix-Marseille University, France
	\item Mauro Maggionni, full professor, Duke University, USA
	%\item Clothilde Mélot, maître de conférence, Aix-Marseille University, France
\end{itemize}

We want to thank Yannick Boursier, Lionel Martin, Johan Paratte, Benjamin Ricaud
and Xavier Bresson who helped us to prepare this proposal and will continue to
provide support should it be retained.

\subsection{Gender parity analysis}

To the best of our knowledge, Fan Chung Graham, Ulrike von Luxburg and Jelena
Kovačević are the only women professors in the field. As for the participants we
cannot comment on every lab but at EPFL the gender parity is comparable to other
events in the field with 20 women for 70 men in the labs of Prof. Vandergheynst,
Frossard, Thiran and Grossglauser.

\subsection{ECTS credits}

We suggest to offer 2 ECTS points to Ph.D. students who did participate in the
school. To that end, a certificate of participation will be delivered at the end
of the week; it is then up to home institutions to award those credits.

\subsection{Location}

With the idea that nice places can help open the mind, we would be delighted to
host this event in the beautiful coastal area of Marseille. It would further be
in contrast to the previous edition which was hosted in Leukerbad, a small Swiss
town up in a valley in the Alps. As Leukerbad, Marseille is a place which is
quiet for the mind and relaxing for the body. 
In addition to its intrinsic attractiveness as a touristic city, Marseille is
easily accessible by plane, which is important for our international audience.
Its closeness to the French Riviera and Paris makes it an even more attractive
place for overseas speakers and participants.

\subsection{Schedule}

The school will last five days, from Monday morning to Friday afternoon. Rooms
will be available from Sunday evening to Friday morning. Participants will have
to arrive and install themselves at CIRM on Sunday as lectures will start Monday
morning. We plan to finish early on Friday and to offer an excursion on
Wednesday afternoon, which leaves us with four and an half days for lectures.
Following the schedule shown in Table~\ref{schedule}, we have room for 24
lessons of 45 minutes. Each of the five main track speakers will be allocated
four lessons and each side track speaker will be allocated one lesson.
Exact allocations will be discussed with the speakers.

Attendees will be offered the opportunity to present their work. Interested
students will have to provide an abstract. The scientific committee will choose
the most relevant ones, which will be allocated a 20 to 30 minutes time slot at
the end of a day. The participation to these presentations is optional.
Additionally, we schedule some free time after lunch to discuss and exchange
ideas about future developments of the field. Each day, we may propose a
different topic for students to brainstorm or debate about.

Depending on funding opportunities and the enthusiasm of the community, notably
at GSP2016\footnote{To be held in May 2016 in Pennsylvania.}, this event may
grow in size (in terms of speakers and attendants) and become a mix of long
tutorial lectures and short lectures. In that case, student presentations may be
morphed into a poster session, but we definitely want students to be able to
present their work.

\begin{table}[ht]
	\centering
	\begin{tabular}{rcrl}
	 8:00 & - &  8:45 & Breakfast \\
	 9:00 & - &  9:45 & Lesson 1 \\
	10:00 & - & 10:45 & Lesson 2 \\
	10:45 & - & 11:15 & Coffee break \\
	11:15 & - & 12:00 & Lesson 3 \\
	12:15 & - & 13:15 & Lunch \\
	13:30 & - & 14:45 & Free time, discussions \\
	15:00 & - & 15:45 & Lesson 4 \\
	16:00 & - & 16:45 & Lesson 5 \\
	16:45 & - & 17:15 & Coffee break \\
	17:15 & - & 18:00 & Lesson 6 \\
	18:30 & - & 19:30 & Student presentations \\
	20:00 &   &       & Dinner and social events
	\end{tabular}
	\caption{Schedule of a typical day.}
	\label{schedule}
\end{table}

\subsection{Dates}

% As this event is targeted at PhD students, who usually take courses and have
% teaching duties, as well as professors who often give lectures in their home
% university, we find it best to organize the school after the spring
% academic semester.
As we aim to co-locate with another summer school, we jointly propose the
following dates (it may be in a different preference order):
\begin{enumerate}
	\setlength{\itemsep}{0pt} \setlength{\parskip}{0pt}
	\item June 5th
	\item May 15th
	\item April 23rd
\end{enumerate}

\subsection{Budget}

The CIRM subsidizes meals and accommodations (in two beds rooms) for 40
participants. As we plan to co-locate with another school, the CIRM would
provide half of his funding to us while the other half will go to the other
event. This will thus cover 20 of our participants. The CIRM package is offered
at 88.5€ per day, which accounts for $88.5 \cdot 5 = 442.50$€ per participant.

Invited speakers will be provided with meals and an individual room at CIRM. A
double room or a flat rate will be offered to those who prefer to self-organize,
e.g. for family reasons. The CIRM prices are: 61.80€ per night for a single room
including breakfast, 15.45€ per meal. We thus budget $61.80 \cdot 5 + 15.45
\cdot 10 = 465.5$€ per speaker. We plan to defray their travel expenses. While
the exact amount will depend on the available funding, we budget 200€ for EU
speakers and 700€ for US speakers

If this proposal is accepted, we will request additional funding to local
communities in the CIRM area\footnote{Coordinated and estimated by Bruno
Torrésani.} and to the Seasonal Schools in Signal Processing (S3P)
Program\footnote{\url{http://signalprocessingsociety.org/seasonal-schools}} of
the IEEE Signal Processing Society (SPS). While there is no call for proposals
for 2018 yet, the winter 2016-2017 version states that ``an SPS contribution of
up to 5k US dollars can be included in the budget''.

To cover additional expenses and to incentivize participants to be ``active'',
we further ask for a registration fee of 200€ per participant from academia and
800€ for non-academic people.

A detailed budget can be found in tables~\ref{tab:expenditures} and
\ref{tab:revenues}.

\begin{table}[ht]
	\centering
	\begin{tabular}{|l|r|r|r|}
	\hline
	Description & Amount (EUR) & Quantity & Expenditures (EUR) \\
	\hline
	Meals and lodging (participants) &  442.50 & 40 & 17'700 \\
	Meals and lodging (speakers)     &  465.50 &  5 &  2'328 \\
	Travel costs for EU speakers     &  200    &  3 &    600 \\
	Travel costs for US speakers     &  700    &  2 &  1'400 \\
	Excursion                        &   30    & 44 &  1'320 \\
	Miscellaneous                    &  502    &  1 &    502 \\
	\hline
	\multicolumn{3}{|l|}{\textbf{Total}} & \textbf{23'850} \\
	\hline
	\end{tabular}
	\caption{Budgeted expenditures.}
	\label{tab:expenditures}
\end{table}

\begin{table}[ht]
	\centering
	\begin{tabular}{|l|r|r|r|}
	\hline
	Description & Amount (EUR) & Quantity & Revenues (EUR) \\
	\hline
	CIRM subsidies        & 442.50 & 20 & 8'850 \\
	IEEE SPS contribution & 5000   &  1 & 5'000 \\
	Local sponsoring      & 2000   &  1 & 2'000 \\
	Registration fees     &  200   & 40 & 8'000 \\
	\hline
	\multicolumn{3}{|l|}{\textbf{Total}} & \textbf{23'850} \\
	\hline
	\end{tabular}
	\caption{Budgeted revenues.}
	\label{tab:revenues}
\end{table}

\printbibliography

\end{document}
